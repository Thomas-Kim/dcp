\documentclass[12pt]{article}

\usepackage{natbib}
\usepackage[cm]{fullpage}
\usepackage{graphicx}
\pagenumbering{gobble}

\linespread{1.00}

\usepackage{titlesec} % Allows customization of titles

\setlength\parindent{24pt}

\usepackage[
  top    = 2.50cm,
  bottom = 1.50cm,
  left   = 1.50cm,
  right  = 1.50cm]{geometry}

\usepackage{setspace}

\title{\vspace{-25mm}\fontsize{16pt}{10pt}\selectfont\textbf{Proposal}} % Article title

\author{
  \textsc{William Morriss} \\
  \textsc{Thomas Kim}
}
\date{}

\begin{document}

\maketitle % Insert title

\section{Overview}
An improved version of 'cp -r,' dcp, was implemented.
This version will takes advantage of Linux's asynchronous IO interfaces to create
opportunities for parallelism, and minimizes the time spent blocking on disk
IO by leveraging fallocate and readahead. In order to ensure reads
are more likely to result in hits in the buffer cache when memory is scarce, POSIX
fadvise is used. To ensure optimal layout of files for writing, POSIX
fallocate is used. The main goal of dcp is to fully saturate disk IO
at all times in order to have a fast, portable solution which supports
any POSIX compliant filesystem interface.

\section{Implementation}

\subsection{Multithreading}

\subsection{Readahead}
Linux readahead and POSIX fadvise provide similar functionality,
so POSIX fadvise will be used.
POSIX fadvise will be used to ensure that pages will be available
in the buffer cache when read, and that pages will be evicted from
the cache as soon as possible. All reads in dcp will
always be sequential, so FADV\_SEQUENTIAL will be used.

\subsection{Fallocate}
POSIX fallocate will be used to allow the destination filesystem to
allocate blocks in an efficient way. The use of fallocate is
fairly straightforward and is expected to increase write speeds
provided the filesystem on the destination disk intelligently
allocates blocks.

\section {Test platform}
\begin{tabular}{l|r r}
  Component & Specification                   & Interface \\
  CPU       & Intel i7-4770k @ 3.50ghz                    \\
  HDD A     &         1tb @ 7200rpm           & SATA 2.0  \\
  HDD B     &                 2tb @ 7200rpm   & SATA 2.0  \\
  SSD A     &         250gb                   & SATA 2.0  \\
  External A& 3tb @ 5400rpm                   & USB 3.0   \\
\end{tabular}
\section{Benchmarking methodology}
Dcp was tested on consumer-grade personal computers using 3 different types
of disk: solid state drive, traditional disk-based hard drive, and
an external USB3.0 disk based hard drive.
Three main benchmarks were run on the data, %TODO
.
Additionally, one practical test was done to demonstrate the advantage
of dcp over standard cp -r. 5.7gb of video files split among 5 subdirectories were copied between
HDD A and HDD B to simulate a backup of videos, a usage pattern common for one of the researchers.
This benchmark was not repeated for other cases as it is simply intended to convey a practical example use
case. %TODO maybe disregard this test case

\section{Results}


\end{document}
