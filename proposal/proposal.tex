\documentclass[12pt]{article}

\usepackage{natbib}
\usepackage[cm]{fullpage}
\usepackage{graphicx}
\pagenumbering{gobble}

\linespread{1.00}

\usepackage{titlesec} % Allows customization of titles

\setlength\parindent{24pt}

\usepackage[
  top    = 2.50cm,
  bottom = 1.50cm,
  left   = 1.50cm,
  right  = 1.50cm]{geometry}

\usepackage{setspace}

\title{\vspace{-25mm}\fontsize{16pt}{10pt}\selectfont\textbf{Proposal}} % Article title

\author{
  \textsc{William Morriss}
  \textsc{Thomas Kim}
}
\date{}

\begin{document}

\maketitle % Insert title
% Timeline Project is due 12/02

\section{Overview}
An improved version of 'cp -r' is proposed.
This version will take advantage of Linux's asynchronous IO interfaces to create
opportunities for parallelism, and will minimize the time spent blocking on disk
IO by leveraging fallocate and readahead. Furthermore, to ensure no requests to
paged out pages in the buffer are blindly initiated, a kernel module to service
requests from dcp for page table entries is proposed.

\section{Key features}
\subsection{Asynchronous I/O}
Dcp will use asynchronous I/O to populate a per-thread in-memory buffer.
Both fixed-size and adaptive-sized buffers will be tested.
For adaptive sizing of the buffer, the size of physical memory
and number of free physical pages will be considered when allocating
the buffer, and can be resized at a regular interval based on system load.
Requests for both reads and writes will be made using the POSIX AIO interface
\cite{manaio}.

\subsection{Adaptive buffer sizing}
As mentioned in the previous section on asynchronous I/O, an adaptive
sizing of the per-thread buffer will be implemented and evaluated for
performance. The combination of increasing the size of the buffer for
larger physical memory sizes and FADV\_SEQUENTIAL, disk idle time could
possibly be reduced by allowing for larger concurrent read requests.
For systems with older and slower disks, as well as directory trees
with a large number of very small files, having a larger buffer
will not necessarily positively affect dcp's efficiency.

\subsection{Multithreading}
Three alternatives for multithreading will be compared for dcp,
threaded synchronous I/O, threaded asynchronous I/O, and unthreaded
asynchronous I/O (\textit{see section 3})

\subsection{Fallocate}
POSIX fallocate will be used to allow the destination filesystem to
allocate blocks in an efficient way. The use of fallocate is
fairly straightforward and is expected to increase write speeds
provided the filesystem on the destination disk intelligently
allocates blocks.

\subsection{Readahead}
POSIX fadvise will be used to ensure that pages will be available
in the buffer cache when read, and that pages will be evicted from
the cache as soon as possible. Both reads and writes in dcp will
generally (\textit{see Page Presence Tracking}) be sequential.

\subsection{Page Presence Tracking}
In order for dcp to be aware of any pages in the per-thread buffer that
are paged out, a kernel module to fetch and return page table entries to
dcp is proposed. Using the page table entries, dcp will have a better
idea of whether the next sequential page in the buffer is in memory,
and can make asynchronous IO requests for pages that are in memory
before page faulting for those pages which were paged out.

\section{Benchmarking methodology}

\section{Timeline}
This project will follow a weekly timeline. \\
\begin{tabular}{|l | l|}
\hline
10/13 & Start \\
10/20 & Naive implementation complete \\
10/27 & MPMC complete \\
11/03 & readahead complete \\
11/10 & fallocate complete \\
11/17 & Testing/optimization \\
11/24 & Benchmarking \\
12/01 & Complete \\
\hline
\end{tabular}


\bibliographystyle{plain}
\bibliography{citations}
\end{document}
